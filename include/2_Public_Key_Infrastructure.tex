\chapter{Public Key Infrastructure}\label{ch:publicKeyInfrastructure}

This chapter explains, how privacy in communication can be achieved and what infrastructure

\section{Public-Key Cryptography}\label{sec:publicKeyCryptography}

"key-management problem with symmetric cryptosystems"

"two different keys - one public and the other private"~\cite{schneier2007applied, diffie1976new}

Public key for encryption, private for decryption.
Private key for signature, public key for signature verification.

To solve the problem of man-in-the-middle attacks, a system of trust is established.
In this system a trusted third party verifies, that the public keys are authentic

\section{Public Key Infrastructure}\label{sec:publicKeyInfrastructure}
please refer to RFC3280

\section{Email Security}\label{sec:emailSecurity}

Security of emails was not a design goal.
Email with it's several protocols is inherently insecure.
Several approaches to secure email, e.g. SMTP over TLS, which only secures traffic but does not provide end-to-end
security.
Truly confidential and authenticated messages only possible with two competing end-to-end cryptographic implementations:
PGP and S/MIME.

\subsection{PGP}\label{subsec:pgp}

Pretty Good Privacy (PGP) was first implemented by Philip Zimmermann in 1991.
Can be used for encryption and signing of data.
See PGP user's guide.

Software support, e.g. in Mozilla Thunderbird via a plugin: Enigmail

OpenPGP, the modern version and de-facto standardized PGP.
Specified in RFC4880.

\subsection{S/MIME}\label{subsec:s/mime}
see RFC5750


\section{Certificates at TUM}\label{sec:certificatesAtTum}

TUM has multiple possible ways of acquiring certificates:

The "Leibniz-Rechenzentrum der Bayerischen Akademie der Wissenschaften" (LRZ in short) offers to sign
certificates~\cite{lrzpki}, which then can be used to secure servers hosted with the LRZ, but also for e-mail security.
The LRZ itself describes the situation as complicated, because of the "many other temporary installed solutions"
("[da] sowohl am LRZ wie auch anderswo provisorische L\"osungen installiert worden sind").

TUM IT-support also offers certification services~\cite{tumZertifikat}.

Additionally, TUMs Faculty of Informatics also runs a registration authority~\cite{inTumCertificates}.

All three of those certification solutions are based on the CA of the Deutsches Forschungsnetz (DFN)~\cite{dfnPki}.
So, in theory any one certification method should have the same "trust", however the certification requirements and
procedures significantly differ.
Furthermore, the process of generating public keys is not handled very well: LRZ provides a guide to generate keys with
the \lstinline{openssl} command line utilities, which disqualifies a lot of the users from using the RA\@.
The Faculty of Informatics generates the keys for their users and stores them secured with a RA generated passphrase.
This fails one of the basic requirements of public-key cryptography, that only the user should ever have access to the
private key.
TUM-IT uses the DFNs webinterface to generate keys, which generates the keys in the users browser and encrypts them with
a user provided passphrase.
This process is a really good compromise between the ability of a layman to generate keys, while not breaking
cryptography guidelines.
However, this webinterface is not very accessible, especially on mobile devices and additionally completely separated
from TUM infrastructure, such that basic integration within TUM workflows or many quality-of-life improvements can't be
made.

