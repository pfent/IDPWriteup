\chapter{Design}\label{ch:design}

In this chapter we describe the approach and design of the of the implemented solution.
First and foremost, a automatic certification process should avoid user interaction whenever possible.

\begin{figure}
    \centering
    \includegraphics[width=\textwidth]{figures/system_architecture.pdf}
    \caption{Architecture of the certificate management system}
    \label{fig:systemArchitecture}
\end{figure}

\section{User Centric features}\label{sec:userDetails}
Certificates require some fields regarding information of the user.
Since most of this information is readily available in organizations user identity management systems, this information
should be used to reduce the interaction the user needs to have with the system.

This identity management can also be used to authenticate the user.
Since most institutions have already verified users identities, those systems can be used as basis of trust.

However, providing a secure way to backup certificates of users is essential to provide a good user experience.

\section{Integration Into Existing Systems}\label{sec:integrationIntoExistingSystems}
User friendliness is not only a design goal of the frontend, but also the backend APIs, that are documented and can be
used intuitively.
Reduction of setup, deployment and development complexity, results in a intuitive for administrators and future
extensions of the system.

Another important factor is the usage of industry standard tools, such as JAVA enterprise application.
Java enterprise servers are often already available in organizations, but can also easily be set up separately.
Different databases in organizations, provide abstraction from concrete database.

\section{Compatability With Embedded Devices}\label{sec:compatabilityWithEmbeddedDevices}
Embedded devices: Hands-off provisioning of certificates.
Fully scriptable generation and distribution of certificates.
This essentially boils down to providing an API, which can be universally used.


\section{Database Connection}\label{sec:databaseConnection}
First task, implementing a generic database connection, that works with any database an organization wishes to use.
Previously only MySQL was supported, a more generalized approach would be better, e.g. to store data in an oracle
database, such as TUM uses for its TUMonline.

Furthermore, mismatches between database schema and java objects are only detected during runtime, which increases
testing overhead.
This can be statically checked at compile time, which should decrease bugs and increase developer productivity

\section{Certificate Exchange}\label{sec:certificateExchange}
Currently lacking.
Several different methods~\cite{hauner2016interoperability}:
LDAP x509 certificate fields in RFC4523~\cite{RFC4523}


An alternative would be DNS-Based Authentication of Named Entities, as defined in RFC8162 via the SMIMEA DNS record
For openpgp via RFC7929, however in this work, we will only concentrate on on X.509 certificates

Most widespread: LDAP, often used in an organization context
