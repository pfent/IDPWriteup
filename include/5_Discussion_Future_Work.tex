\chapter{Discussion and Future Work}\label{ch:futureWork}

In this chapter, we

\section*{Discussion}
% TODO: wie das neue System den Status Quo verbessert

\subsection*{Limitations regarding certificate authorities}
Most probably most organization's have sufficient identity checks when people join it should already comply with
relatively strict identification rules.
However for our example at TUM, the situation is a bit more complicated:
The DFN provides publicly available guidelines\footnote{https://www.pki.dfn.de/fileadmin/PKI/DFN-PKI_CP.pdf} how to
handle certificates.
For email certificates it requires the \say{Authentification of a natural person} (cf.\ Section 3.2.3 in the linked
document).
This authentication needs to happen in-person (\say{persönliche Identitätsprüfung}), but not necessarily at the time of
certification.

An additional requirement for certification is the proper documentation of this procedure, which might need to be
checked against current employment and matriculation practices at TUM\@.
Most probably additional organizational convincing needs to happen, that a proposed \say{transfer} of checked identity
can indeed happen via secure identity management systems within TUM\@.

\section*{Future Work}
To successfully implement the results, we and the working group \say{secure email} achieved, there are still a lot of
open tasks.
To finish our work on the certificate management component, an actual deployment of the system needs to happen.
Foundations for a proper long term deployment were layed in our work, but to deploy the actual system,
an actual certificate authority, which is trusted by arbitrary clients needs to be convinced to endorse our system.

Integration into existing infrastructure at TUM would also be desirable, e.g.\ into the CAMPUSonline instance, which
should allow to run Java applications and view the web frontend in a familiar environment.

To convince a certificate authority to trust our system, the security of our implementations should be audited
separately.
Since this system is critical for user's trust and freedom, no corners should be cut.
Our work's focus was not the security of the application and questions like: \say{What fields in a user's certificates
are trusted and which can be misused?}, were not answered in this work.

Last but not least, substantial work needs to be put into user facing clients.
Even though most clients already support email encryption, configuration is still a hassle, especially for the
non-technical users.
To achieve a true privacy-by-default system, addons, that automatically set-up and configure all email clients should be
developed.
A special hurdle for for future work will be web based email clients, even though they could potentially have the best
privacy-by-default behaviour.
