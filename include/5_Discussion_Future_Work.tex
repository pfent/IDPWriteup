\chapter{Discussion and Future Work}\label{ch:futureWork}

In this chapter, we discuss opportunities and limitations of our implementation when applied to be used at TUM.
We also give an outlook on how our system can be used in production and extended for future applications.

\section*{Discussion}
% TODO: wie das neue System den Status Quo verbessert

For TUM our implementation would be a huge improvement with regards to usability and accessibility.
For majority of use cases, consolidate and replace existing systems.
This has the advantage to make the process simpler, with less possible causes for confusion for the average user.
Since we can also eliminate most of the manual interaction, users can generate certificates much faster, which should
increase encryption usage similar to how Let's Encrypt did for HTTPS usage.

Additionally, we provide a interface, that is also comfortably usable on mobile devices, which was previously not
available.
We also provide users with a convenient and secure way to transfer private keys to their mobile devices.
As many users see their mobile devices as the most prominent way to communicate this should have been a focus long ago.

Security considerations:
JS being delivered by backend.
Also contains the code, that generates the private keys.
Would enable the possibility for attacks, where the server delivers malicious code to the client, which then could
compromise private keys.

First counter-measure against man-in-the-middle attacks is HTTPS, so an attacker would need to compromise the server
delivering the JS.
Additionally, we provide the possibility to not use JS generated private keys, but only use the system to
submit CSRs.
Not ideal, since this is not a recommendable way for average users.

The most ideal counter-measure would be to have a fully reproducible build of the open-source
frontend\footnote{\url{https://reproducible-builds.org/}}.
This would allow anyone to independently audit the code and be sure the server only delivers that reproducibly built
frontend.
The security-aware users then only would need to compare a checksum of the code to be sure to execute legitimate code.

\subsection*{Limitations regarding certificate authorities}
Most probably most organization's have sufficient identity checks when people join it should already comply with
relatively strict identification rules.
However for our example at TUM, the situation is a bit more complicated:
The DFN provides publicly available guidelines\footnote{\url{https://www.pki.dfn.de/fileadmin/PKI/DFN-PKI_CP.pdf}} how
to handle certificates.
For email certificates it requires the \say{Authentification of a natural person} (cf.\ Section 3.2.3 in the linked
document).
This authentication needs to happen in-person (\say{persönliche Identitätsprüfung}), but not necessarily at the time of
certification.

An additional requirement for certification is the proper documentation of this procedure, which might need to be
checked against current employment and matriculation practices at TUM\@.
Most probably additional organizational convincing needs to happen, that a proposed \say{transfer} of checked identity
can indeed happen via secure identity management systems within TUM\@.

\section*{Future Work}
To successfully implement the results, we and the working group \say{secure email} achieved, there are still a lot of
open tasks.
To finish our work on the certificate management component, an actual deployment of the system needs to happen.
Foundations for a proper long term deployment were layed in our work, but to deploy the actual system,
an actual certificate authority, which is trusted by arbitrary clients needs to be convinced to endorse our system.

Integration into existing infrastructure at TUM would also be desirable, e.g.\ into the CAMPUSonline instance, which
should allow to run Java applications and view the web frontend in a familiar environment.

To convince a certificate authority to trust our system, the security of our implementations should be audited
separately.
Since this system is critical for user's trust and freedom, no corners should be cut.
Our work's focus was not the security of the application and questions like: \say{What fields in a user's certificates
are trusted and which can be misused?}, were not answered in this work.

Last but not least, substantial work needs to be put into user facing clients.
Even though most clients already support email encryption, configuration is still a hassle, especially for the
non-technical users.
To achieve a true privacy-by-default system, addons, that automatically set-up and configure all email clients should be
developed.
A special hurdle for for future work will be web based email clients, even though they could potentially have the best
privacy-by-default behaviour.
