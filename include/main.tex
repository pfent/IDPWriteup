\subfile{include/0_Intro.tex}

\subfile{include/1_Related_Work.tex}

\subfile{include/2_Public_Key_Infrastructure.tex}

\subfile{include/3_Design.tex}

\subfile{include/4_Implementation.tex}

\chapter{Conclusion}\label{ch:conclusion}
% TODO
Encrypting email is very important.

Enabling users to use encryption requires tight integration into existing workflows.

Requirements for a DFN RA are already met, since all users of TUM's identity management have their identity checked.
Employees need to show their passport when applying for a contract.
Students need to upload a copy of their passport when applying for their program.
The only reasons blocking a roll-out of an automatic and universal certificate management system are or political
nature.

The CMS should also generalize very well to arbitrary organizations, since all required components for integration are
very common.
User authentication in the vast majority of organizations is covered by support for LDAP and Active Directory.
Automatic certificate approval be covered by requirements to check identities, that are usually present for health
insurances, either for employees or at the moment of matriculation.

\chapter{Future Work}\label{ch:futureWork}

Future work includes actual deployment of the system.
This requires acquisition of an actual certificate authority, that is trusted by clients.

Integration into existing infrastructure at TUM also needs to happen, e.g.\ into the CAMPUSonline instance, which should
allow to run JAVA applications

And last but not least the security of the implementations should be independently audited.
Since this system is critical for users trust and freedom, no corners should be cut.
Focus of this work was not the security of the application and questions like, what fields in users certificates are
allowed, were not answered in this work.
