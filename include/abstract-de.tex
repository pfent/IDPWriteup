\small

Eines der größten Probleme von E-Mail Kommunikation ist, dass diese fast ausschließlich unverschlüsselt abläuft.
Die Erstellung von E-Mail-Verschlüsselungs-Zertifikaten ist meistens nur schlecht in Unternehmensabläufe eingebunden und
daher für Nutzer nur schlecht zugänglich.
Da alle Voraussetzungen für starke Verschlüsselung zumindest bei interner E-Mail Kommunikation gegeben sind, sollte
diese auch standardmäßig im Einsatz sein.
Um dies zu ermöglichen implementieren wir in unserer Arbeit ein Zertifikatsverwaltungssystem, das stark in bestehende
Unternehmensabläufe integriert ist.

In unserer Arbeit portierten wir schon vorhandener Code zur Zertifikatsverwaltung wurde auf schon existierende
Infrastruktur.
Danach integrierten wir die Nutzer- und Zertifikatsverwaltung in bestehende Verzeichnisdienste und automatisierten alle
nutzerseitigen Prozesse weitestgehend.
Das Resultat unserer Arbeit ist ein funktionierendes, experimentelles, Web-basiertes Zertifikatsverwaltungssystem, das
in der Lage ist automatisch Zertifikate für Benutzer der TUM zu erzeugen.
Obwohl für den allgemeinen Einsatz noch eine vertrauenswürdige Zertifizierungsstelle fehlt, funktioniert das System
bereits mit selbst signierten Zertifikaten und hilft Nutzern einfacher E-Mails zu verschlüsseln.
