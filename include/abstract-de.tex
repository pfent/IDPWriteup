\small

Eines der größten Probleme von E-Mail Kommunikation ist, dass diese fast ausschließlich unverschlüsselt abläuft.
Die Erstellung von E-Mail-Verschlüsselungs Zertifikaten ist meißtend nur schlecht in Unternehmensabläufe eingebunden und
daher für Nutzer nur schlecht zugänglich.
Da alle Vorraussetzungen für starke Verschlüsselung, zumindest bei interner E-Mail Kommunikation gegeben sind, sollte
diese standardmäßig aktiviert sein.
Um dies zu ermöglichen, implementieren wir in unserer Arbeit ein Zertifikatsverwaltungssystem, das stark in bestehende
Unternehmensabläufe integriert ist.

Die hauptsächlichen Funktionen, die in dieser Arbeit erstellt wurden sind:
\begin{itemize}
    \item Den schon vorhandenen Code zur Zertifikatsverwaltung für schon existierende Infrastruktur zu portieren
    \item Die Nutzer- und Zertifikatsverwaltung in bestehende Verzeichnisdienste zu integrieren
    \item Die Benutzerseitigen Prozesse weitestgehend zu automatisieren
\end{itemize}

Das Resultat unserer Arbeit is ein funktionierendes, experimentelles Web basiertes Zertifikatsverwaltungssystem, das in
der Lage ist automatisch Zertifikate für Benutzer der TUM zu erzeugen.
Obwohl für den allgemeinen Einsatz noch eine vertrauenswürdige Zertifizierungsstelle fehlt, fuktioniert das System
bereits mit selbst signierten Zertifikaten und erlaubt es Nutzern einfacher E-Mails zu verschlüsseln.
