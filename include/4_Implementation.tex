\chapter{Implementation}\label{ch:implementation}
In this chapter, we describe the process of applying the design as described in Chapter~\ref{ch:design} to the
Certificate Management System.

\section{Exising Prototype}\label{sec:exisingPrototype}
Dr.\ Wachs et al.\ already implemented a prototype, that can be used to issue and manage certificates.
This system can be used as basis of our work, however several key features are still lacking:

After analyzing the prototype, it seemed overly complex and overly modularized.
Several abstractions don't make sense, i.e.\ it was designed to support different so-called "Runtime Abstraction
Layers", while none of that abstraction was needed.
Furthermore, the project was on first sight modularized with several "common" modules, which turned out to be tightly
coupled into the whole system.
This made things worse than no modularization at all, since related classes were distributed between modules and
functionality was duplicated between modules.

Also the deployment of the software was rather fragile and consisting of a combination of bash and maven files and
separate database configuration instructions, which needed to be executed in the right order.
This resulted in a scary deployment process, which resulted in a virtual machine, which was updated every once in a
while.
This is obviously a bad idea, since any "Runtime Abstraction" prior to that machine was rendered useless.

Furthermore, when looking at the JAVA code, no clear lines of input validation were clear.
Almost every class had their own nullpointer checks, and returning null themselves in error cases and only logging the
error message.
This resulted in the vast majority of unit tests being dedicated to nullability checks.

Lastly, almost all user management features were not included in the overall design of the system, which already
struggled under its own complexity, while lacking some of the most basic features.

\section{Complexity Reduction}\label{sec:complexityReduction}
% TODO
No RAL, only a single Spring Boot Application

Bundeled competences with reduced modularization: Certificate Management System and REST-API

\section{Deployment With Gradle}\label{sec:deploymentWithGradle}
% TODO: proof read

Previous build automation system with Apache Maven.
Slow build times and unflexible extensibility, which resulted in separate configuration scripts, which needed to be
executed in the right order.
Integration of modern web development frameworks for the frontend also only available for alternative systems, would
result in additional configuration effort.
This should obviously be handled via the build system.
One commonly used modern alternative is Gradle, which we used to reduce the need for separate scripts, since Gradle can
be used to perform arbitrary tasks using the Apache Groovy language.

Updating dependencies should be easy, especially for a security-critical application like ours.
More than 25 different third-party libraries used for this project.
Updating those dependencies is necessary for security updates, especially in a security oriented application, handling
sensitive information of users and responsible to secure certificates.

Outdated dependencies can be detected via the "gradle-versions-plugin", which displays the newest available versions
from open-source repositories. %TODO: example output of that plugin?

While handling the dependency updates, we noticed problems with updating individual dependencies, since the dependencies
have internal dependencies themselves, commonly described as "Dependency Hell": %https://en.wikipedia.org/wiki/Dependency_hell

Especially time consuming and frustrating, since mismatched versions of the Spring Framework and dependent dependencies
can lead to "ClassNotFoundException"s at runtime.
This is inherently due to Javas implementation of lazily loading classes~\cite{gosling2014java}.

Our implementation solves this, by specifying version numbers globally, which removes the possibility of "missing" to
update one dependent version.
This couples compatible versions of the Spring Framework and Log4J together which should automatically keep the versions
compatible.

\section{Database Versioning And Migrations}\label{sec:databaseVersioningAndMigrations}
While working on the database, we quickly noticed, that the database access needs severe improvement.
Since every change in the database schema resulted in multiple changes in the breakages, all of which only happened at
runtime.
Since none of the existing tests actually verified the database, this only manifested at runtime in hard to debug
crashes.
Also, no good upgrade path is available, since there is no no versioning information available in the database.
The jackhammer method of exporting the whole database and recreating it each time the schema is updated was quickly
discarded.

Instead, we decided to use two industry standard dependencies: jOOQ~\cite{jooq} and flyway~\cite{flyway}

The idea behind jOOQ is static class generation to check matching models at compile time and features an alternative
Domain Specific Language (DSL) to SQL\@.
This idea provides several advantages over classic Java Database Connectivity (JDBC) in combination with raw SQL:
\begin{itemize}
    \item Type safety between the database model and the Java application
    \item Support for different databases
    \item Enterprise support if needed for enterprise databases
\end{itemize}

Flyway provides automate creation of database schema and migration of old data via a gradle task, that can be defined to
automatically apply the correct migrations for the currently deployed version.
This severely reduces the work, that needs to be done when updating the database schema.

All in all, this makes the future development of the application much easier, since for each change to the database,
only a single migration needs to be written.
Subsequently, the matching database models are generated and any mismatches are immediately reported as compiler errors.

\section{Core Features}\label{sec:coreFeatures}
User authentication with LDAP.
Two different user information approaches:
Or according to Microsofts Active Directory
Both of which are handled via the LDAP protocol

Publishing of certificates can happen via those implementations, i.e.\ publishing of RFC4519~\cite{RFC4519} compliant exchange between
user information.
